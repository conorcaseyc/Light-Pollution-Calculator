\section{Review of Results}
The aim of this project was to establish a mathematical model that could predict photopollution (the dependent variable) solely using population density (independent variable), and if we were able to create such a mathematical model, could we determine the severity of photopollution in each particular area.

Our findings suggests there is a relationship between the two variables. The relationship between the two variables is described in the equation \ref{modelrevisit}.

\begin{equation}
\label{modelrevisit}
    I = 0.03510413 d_x - 14.32180871 \tag{\ref{model} revisited}
\end{equation}

Therefore, we have an established an empirical relationship between the two variables. An empirical relationship is a relationship, or correlation that is supported by experiment and observation, but not necessarily supported by theory.\cite{empirical} The correlation coefficient is 0.916. Our correlation coefficient shows a strong positive correlation between photopollution and population density. This means that when population density increases, photopollution tends to increase. In comparison to Walker's law, we found a correlation coefficient of 0.61. Therefore, this suggests that there is a stronger correlation between population density and photopollution and that our findings support the conclusion that population density is a more reliable predictor of photopollution. Our findings can be summed up with figure \ref{qgis}, and figure \ref{ireland}. Figure \ref{pdheatmap}, is our generated population density heat map of Ireland, figure \ref{photopollutionmap}, is our generated photopollution map, and figure \ref{ireland} is a nighttime survey of Ireland. The figures are striking, and are practically identical. These figures beautiful prove our model's accuracy and the correlation between photopollution and population density. 

The mean of the photopollution value from all the data we collected was 31.2 LUX and the standard deviation was approximately 22.822. This suggests that a photopollution value below approximately 8.378 LUX, and above 54.022 LUX produced in a town is outside the norm. This is most likely an accurate representation of the photopollution produced in towns in the providence of Munster, as all towns in Munster fall between a population density of 600 and 2000 people per square kilometre (with the notable exception of Cork City). Therefore, we covered a wide spread of different values within this range. 

\section{Looking Back}
Looking back, while there may be a strong positive statistical correlation between the two variable, there is a few possible sources of error:

\begin{itemize}
\item The root of the equation \ref{modelrevisit} is at approximately 408.029416909. This means that this model is ineffective with population densities less than 409 people per $km^{2}$. A exponential equation would improve more effective, however, we did not have enough data to formulate such an equation.

\item While towns in Munster should have similar street lighting systems due to their close proximity, and their similar appearances on nighttime survey maps as provided by the VIIRS 2018 (March) Radiance, this does not necessarily mean these towns employ similar lighting systems.

\item During the data collection, the moon was gradually progressing towards a half moon. While this may not have had a large impact on the photopollution value collected from each site as the values were collected during the waxing crescent phase, there is certainly a possibly that it could have interfered slightly.

\item The weather in South Tipperary was partially cloudy. This could have possibly interfered with the measurements taken at the zenith at these locations. While weather was an issue in some instances, the days we chose, were the best possible days to carry out the experiment based on weather predictions from the astronomy forecast provided by Accuweather. 

\item The photometre app had a margin of error of approximately plus or minus five percent. While this is definitely not a terrible margin of error, there is certainly room for improvement. We did attempt to mitigate this error by using a handheld photometer, however, it proved to be incompatible for use with a telescope. The margin of error of said photometer, was approximately plus or minus 4 percent, therefore, possible changes in results would have been negligible.    

\item All population density data used throughout the course of this project was provided by the Central Statistics Office. Their latest population density figures were taken during the 2016 Census. Since 2016, population density figures of the sites we gathered data from could have deviated from their recorded values in 2016; therefore, this could slightly affect the accuracy and validity of our model. That being said, however, it has only been about two years since said Census, therefore, any changes in population density would be small, and as a result, would have a negligible impact on our results.

\item We cannot conclude that one of the variables is the cause of the other. A lurking variable, such as Gross Domestic Product, lighting systems, etc. could be responsible for the increase in photopollution (the dependent variable), i.e. \textbf{correlation does not always imply causality.}\cite{am4}
\end{itemize}

\section{Looking Ahead}
Looking ahead, there are a number of improvements we plan on implementing. 

\begin{itemize}
\item We would like to collect more data from various different sites, especially data that is currently on the extreme ends of our current sample set, i.e. population densities below 600 people per $km^{2}$, and above 2,000 people per $km^{2}$. We would also like to incorporate sites outside the providence of Munster, and when measuring new sites, we would like to use a photometer with an margin of error less than approximately plus or minus two percent. All of these additions would further test our model's accuracy, and will reduce the possible sources of error in our experimental method. This would also allow us to look into the possibility of an exponential equation better representing the relationship between photopollution and population density, as we would have a larger data set to deal with.  
\item As mentioned previously, all present models are currently based on Walker's Law, including fairly precise models, such as, Garstang’s Model.\cite{walkerlaw} These models make use what is called radiative transfer (the propagation of light rays through a medium together with absorption \& scattering). This can provide a more realistic model for areas close to the emitting source. However, in comparison to Walker's Law, we believe, and have shown that population density is a more reliable predictor of photopollution in any particular location. Therefore, we plan on incorporating various aspects of existing models, such as radiative transfer, into our model. We believe this would greatly enhance its accuracy, and vastly improve its ability to predict stargazing conditions accurately. 
\end{itemize}